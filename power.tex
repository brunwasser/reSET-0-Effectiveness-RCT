% Options for packages loaded elsewhere
\PassOptionsToPackage{unicode}{hyperref}
\PassOptionsToPackage{hyphens}{url}
%
\documentclass[
]{article}
\usepackage{lmodern}
\usepackage{amssymb,amsmath}
\usepackage{ifxetex,ifluatex}
\ifnum 0\ifxetex 1\fi\ifluatex 1\fi=0 % if pdftex
  \usepackage[T1]{fontenc}
  \usepackage[utf8]{inputenc}
  \usepackage{textcomp} % provide euro and other symbols
\else % if luatex or xetex
  \usepackage{unicode-math}
  \defaultfontfeatures{Scale=MatchLowercase}
  \defaultfontfeatures[\rmfamily]{Ligatures=TeX,Scale=1}
\fi
% Use upquote if available, for straight quotes in verbatim environments
\IfFileExists{upquote.sty}{\usepackage{upquote}}{}
\IfFileExists{microtype.sty}{% use microtype if available
  \usepackage[]{microtype}
  \UseMicrotypeSet[protrusion]{basicmath} % disable protrusion for tt fonts
}{}
\makeatletter
\@ifundefined{KOMAClassName}{% if non-KOMA class
  \IfFileExists{parskip.sty}{%
    \usepackage{parskip}
  }{% else
    \setlength{\parindent}{0pt}
    \setlength{\parskip}{6pt plus 2pt minus 1pt}}
}{% if KOMA class
  \KOMAoptions{parskip=half}}
\makeatother
\usepackage{xcolor}
\IfFileExists{xurl.sty}{\usepackage{xurl}}{} % add URL line breaks if available
\IfFileExists{bookmark.sty}{\usepackage{bookmark}}{\usepackage{hyperref}}
\hypersetup{
  pdftitle={Power Analysis},
  hidelinks,
  pdfcreator={LaTeX via pandoc}}
\urlstyle{same} % disable monospaced font for URLs
\usepackage[margin=1in]{geometry}
\usepackage{color}
\usepackage{fancyvrb}
\newcommand{\VerbBar}{|}
\newcommand{\VERB}{\Verb[commandchars=\\\{\}]}
\DefineVerbatimEnvironment{Highlighting}{Verbatim}{commandchars=\\\{\}}
% Add ',fontsize=\small' for more characters per line
\usepackage{framed}
\definecolor{shadecolor}{RGB}{248,248,248}
\newenvironment{Shaded}{\begin{snugshade}}{\end{snugshade}}
\newcommand{\AlertTok}[1]{\textcolor[rgb]{0.94,0.16,0.16}{#1}}
\newcommand{\AnnotationTok}[1]{\textcolor[rgb]{0.56,0.35,0.01}{\textbf{\textit{#1}}}}
\newcommand{\AttributeTok}[1]{\textcolor[rgb]{0.77,0.63,0.00}{#1}}
\newcommand{\BaseNTok}[1]{\textcolor[rgb]{0.00,0.00,0.81}{#1}}
\newcommand{\BuiltInTok}[1]{#1}
\newcommand{\CharTok}[1]{\textcolor[rgb]{0.31,0.60,0.02}{#1}}
\newcommand{\CommentTok}[1]{\textcolor[rgb]{0.56,0.35,0.01}{\textit{#1}}}
\newcommand{\CommentVarTok}[1]{\textcolor[rgb]{0.56,0.35,0.01}{\textbf{\textit{#1}}}}
\newcommand{\ConstantTok}[1]{\textcolor[rgb]{0.00,0.00,0.00}{#1}}
\newcommand{\ControlFlowTok}[1]{\textcolor[rgb]{0.13,0.29,0.53}{\textbf{#1}}}
\newcommand{\DataTypeTok}[1]{\textcolor[rgb]{0.13,0.29,0.53}{#1}}
\newcommand{\DecValTok}[1]{\textcolor[rgb]{0.00,0.00,0.81}{#1}}
\newcommand{\DocumentationTok}[1]{\textcolor[rgb]{0.56,0.35,0.01}{\textbf{\textit{#1}}}}
\newcommand{\ErrorTok}[1]{\textcolor[rgb]{0.64,0.00,0.00}{\textbf{#1}}}
\newcommand{\ExtensionTok}[1]{#1}
\newcommand{\FloatTok}[1]{\textcolor[rgb]{0.00,0.00,0.81}{#1}}
\newcommand{\FunctionTok}[1]{\textcolor[rgb]{0.00,0.00,0.00}{#1}}
\newcommand{\ImportTok}[1]{#1}
\newcommand{\InformationTok}[1]{\textcolor[rgb]{0.56,0.35,0.01}{\textbf{\textit{#1}}}}
\newcommand{\KeywordTok}[1]{\textcolor[rgb]{0.13,0.29,0.53}{\textbf{#1}}}
\newcommand{\NormalTok}[1]{#1}
\newcommand{\OperatorTok}[1]{\textcolor[rgb]{0.81,0.36,0.00}{\textbf{#1}}}
\newcommand{\OtherTok}[1]{\textcolor[rgb]{0.56,0.35,0.01}{#1}}
\newcommand{\PreprocessorTok}[1]{\textcolor[rgb]{0.56,0.35,0.01}{\textit{#1}}}
\newcommand{\RegionMarkerTok}[1]{#1}
\newcommand{\SpecialCharTok}[1]{\textcolor[rgb]{0.00,0.00,0.00}{#1}}
\newcommand{\SpecialStringTok}[1]{\textcolor[rgb]{0.31,0.60,0.02}{#1}}
\newcommand{\StringTok}[1]{\textcolor[rgb]{0.31,0.60,0.02}{#1}}
\newcommand{\VariableTok}[1]{\textcolor[rgb]{0.00,0.00,0.00}{#1}}
\newcommand{\VerbatimStringTok}[1]{\textcolor[rgb]{0.31,0.60,0.02}{#1}}
\newcommand{\WarningTok}[1]{\textcolor[rgb]{0.56,0.35,0.01}{\textbf{\textit{#1}}}}
\usepackage{graphicx,grffile}
\makeatletter
\def\maxwidth{\ifdim\Gin@nat@width>\linewidth\linewidth\else\Gin@nat@width\fi}
\def\maxheight{\ifdim\Gin@nat@height>\textheight\textheight\else\Gin@nat@height\fi}
\makeatother
% Scale images if necessary, so that they will not overflow the page
% margins by default, and it is still possible to overwrite the defaults
% using explicit options in \includegraphics[width, height, ...]{}
\setkeys{Gin}{width=\maxwidth,height=\maxheight,keepaspectratio}
% Set default figure placement to htbp
\makeatletter
\def\fps@figure{htbp}
\makeatother
\setlength{\emergencystretch}{3em} % prevent overfull lines
\providecommand{\tightlist}{%
  \setlength{\itemsep}{0pt}\setlength{\parskip}{0pt}}
\setcounter{secnumdepth}{-\maxdimen} % remove section numbering

\title{Power Analysis}
\author{}
\date{\vspace{-2.5em}}

\begin{document}
\maketitle

{
\setcounter{tocdepth}{1}
\tableofcontents
}
\hypertarget{notes}{%
\section{Notes}\label{notes}}

As of Feb 18, 2021, I have only run models assuming between-cluster
variance of about 10\%. I am guessing that this is a high estimate.
Also, I have assumed that we will have all 300 participants in both arms
with the outcome variable not missing. Future analyses should consider
other estimates of between-cluster variance and should evaluate the
potential impact of attrition.

\hypertarget{prepare-workspace}{%
\section{Prepare workspace}\label{prepare-workspace}}

Load the required packages.

\begin{Shaded}
\begin{Highlighting}[]
\KeywordTok{require}\NormalTok{( clusterPower )}
\KeywordTok{require}\NormalTok{( longpower )}
\end{Highlighting}
\end{Shaded}

\hypertarget{gee-model}{%
\section{GEE Model}\label{gee-model}}

\hypertarget{gee-model-assuming-or0.50}{%
\subsection{GEE Model Assuming
OR=0.50}\label{gee-model-assuming-or0.50}}

Estimate power for rejecting the null hypothesis that difference in the
odds of non-abstinence (0=abstinent; 1=used drugs) across the reSET-O
and treatment as usual care (TAU) conditions is 0 at weeks 9-12 of the
study. The model assumes the following:

\begin{itemize}
\tightlist
\item
  Random allocation of half the total cluster (k=6) to reSET-O and half
  (k=6) to TAU
\item
  Equal cluster sizes of 50 patients (300 patients in each group)
\item
  Probability of non-abstinence in TAU assumed to be 40\% based on
  \href{https://www.tandfonline.com/doi/full/10.1080/03007995.2020.1846022}{Maricich
  et al., 2020}
\item
  Probability of non-abstinence in reSET-O assumed to be 25\% based on
  \href{https://www.tandfonline.com/doi/full/10.1080/03007995.2020.1846022}{Maricich
  et al., 2020}
\item
  This comes to an assumed odds ratio (OR) = 0.5, a 50\% reduction in
  odds.
\end{itemize}

\begin{Shaded}
\begin{Highlighting}[]
\CommentTok{# geesim1 <- cps.binary( nsim = 1000,}
\CommentTok{#                        nsubjects = 50, # 50 participants per cluster assuming equal sizes}
\CommentTok{#                        nclusters = 6, # 6 clusters per treatment arm}
\CommentTok{#                        p1=.40, # 40% non-abstinence rate in TAU in Maricich et al. (2020)}
\CommentTok{#                        p2=.25, # 25% non-abstinence rate in reSET-0 group in Maricich et al. (2020)}
\CommentTok{#                        sigma_b_sq=0.024, # variance = p(1-p)=0.24; assume 10% variance is between)}
\CommentTok{#                        sigma_b_sq2=0.019, # variance = p(1-p)=0.1875; assume 10% variance is between)}
\CommentTok{#                        alpha=.05,}
\CommentTok{#                        method='gee',}
\CommentTok{#                        quiet=F,}
\CommentTok{#                        seed=0218211,}
\CommentTok{#                        lowPowerOverride = TRUE)}
\KeywordTok{load}\NormalTok{( }\StringTok{'geesim1.RData'}\NormalTok{ )}
\KeywordTok{summary}\NormalTok{( geesim1 )                       }
\end{Highlighting}
\end{Shaded}

\begin{verbatim}
## 
## Monte Carlo Power Estimation based on 1000 Simulations: Simple Design, Binary Outcome. Note: 0 additional models were fitted to account for non-convergent simulations.
## 
## Power Estimate (alpha = 0.05):
##  Power Lower.95.CI Upper.95.CI Alpha  Beta Converged Requested
##  0.956   0.9413796   0.9678505  0.05 0.044      1000      1000
## 
## Method: Generalized Estimating Equation 
## 
## Variance Parameters:
##      sigma_b_sq
## Arm1      0.024
## Arm2      0.019
## 
## Clusters:
##      n.clust
## Arm1       6
## Arm2       6
## 
## Observations:
## $Arm1
## [1] 50 50 50 50 50 50
## 
## $Arm2
## [1] 50 50 50 50 50 50
## 
## 
## Convergence:
## TRUE 
## 1000
\end{verbatim}

This simulation suggests that our power would be between 0.94-0.97 --
i.e., well powered.

\hypertarget{gee-model-assuming-or0.80}{%
\subsection{GEE Model Assuming
OR=0.80}\label{gee-model-assuming-or0.80}}

Now we will calculate power using a simulation assuming a more modest
effect where reSET-O results in a 20\% reduction in the odds.

\begin{itemize}
\tightlist
\item
  Assume a 40\% non-abstinence rate in TAU
\item
  Assume a 35\% non-abstinence rate in reSET-O
\item
  Keep all of the rest of the assumptions the same as in the previous
  simulation
\end{itemize}

\begin{Shaded}
\begin{Highlighting}[]
\CommentTok{# geesim2 <- cps.binary( nsim = 1000,}
\CommentTok{#                        nsubjects = 50, # 50 participants per cluster assuming equal sizes}
\CommentTok{#                        nclusters = 6, # 6 clusters per treatment arm}
\CommentTok{#                        p1=.40, # 40% non-abstinence rate in TAU in Maricich et al. (2020)}
\CommentTok{#                        p2=.35, # 25% non-abstinence rate in reSET-0 group in Maricich et al. (2020)}
\CommentTok{#                        sigma_b_sq=0.024, # variance = p(1-p)=0.24; assume 10% variance is between)}
\CommentTok{#                        sigma_b_sq2=0.023, # variance = p(1-p)=0.2275; assume 10% variance is between)}
\CommentTok{#                        alpha=.05,}
\CommentTok{#                        method='gee',}
\CommentTok{#                        quiet=F,}
\CommentTok{#                        seed=0218211,}
\CommentTok{#                        lowPowerOverride = TRUE )}
\KeywordTok{load}\NormalTok{( }\StringTok{'geesim2.RData'}\NormalTok{ )}
\KeywordTok{summary}\NormalTok{( geesim2 )                       }
\end{Highlighting}
\end{Shaded}

\begin{verbatim}
## 
## Monte Carlo Power Estimation based on 1000 Simulations: Simple Design, Binary Outcome. Note: 0 additional models were fitted to account for non-convergent simulations.
## 
## Power Estimate (alpha = 0.05):
##  Power Lower.95.CI Upper.95.CI Alpha  Beta Converged Requested
##  0.263   0.2359435   0.2914594  0.05 0.737      1000      1000
## 
## Method: Generalized Estimating Equation 
## 
## Variance Parameters:
##      sigma_b_sq
## Arm1      0.024
## Arm2      0.023
## 
## Clusters:
##      n.clust
## Arm1       6
## Arm2       6
## 
## Observations:
## $Arm1
## [1] 50 50 50 50 50 50
## 
## $Arm2
## [1] 50 50 50 50 50 50
## 
## 
## Convergence:
## TRUE 
## 1000
\end{verbatim}

This simulation suggests that our power would be between 0.23-0.29 --
i.e., poorly powered.

\hypertarget{gee-model-assuming-or0.65}{%
\subsection{GEE Model Assuming
OR=0.65}\label{gee-model-assuming-or0.65}}

Now we will calculate power using a simulation assuming an in-between
effect where reSET-O results in a \textasciitilde35\% reduction in the
odds.

\begin{itemize}
\tightlist
\item
  Assume a 40\% non-abstinence rate in TAU
\item
  Assume a 30\% non-abstinence rate in reSET-O
\item
  Keep all of the rest of the assumptions the same as in the previous
  simulation
\end{itemize}

\begin{Shaded}
\begin{Highlighting}[]
\CommentTok{# geesim3 <- cps.binary( nsim = 1000,}
\CommentTok{#                        nsubjects = 50, # 50 participants per cluster assuming equal sizes}
\CommentTok{#                        nclusters = 6, # 6 clusters per treatment arm}
\CommentTok{#                        p1=.40, # 40% non-abstinence rate in TAU in Maricich et al. (2020)}
\CommentTok{#                        p2=.30, # 25% non-abstinence rate in reSET-0 group in Maricich et al. (2020)}
\CommentTok{#                        sigma_b_sq=0.024, # variance = p(1-p)=0.24; assume 10% variance is between)}
\CommentTok{#                        sigma_b_sq2=0.021, # variance = p(1-p)=0.21; assume 10% variance is between)}
\CommentTok{#                        alpha=.05,}
\CommentTok{#                        method='gee',}
\CommentTok{#                        quiet=F,}
\CommentTok{#                        seed=0218211,}
\CommentTok{#                        lowPowerOverride = TRUE )}
\KeywordTok{load}\NormalTok{( }\StringTok{'geesim3.RData'}\NormalTok{ )}
\KeywordTok{summary}\NormalTok{( geesim3 )            }
\end{Highlighting}
\end{Shaded}

\begin{verbatim}
## 
## Monte Carlo Power Estimation based on 1000 Simulations: Simple Design, Binary Outcome. Note: 0 additional models were fitted to account for non-convergent simulations.
## 
## Power Estimate (alpha = 0.05):
##  Power Lower.95.CI Upper.95.CI Alpha  Beta Converged Requested
##  0.688   0.6582616   0.7166268  0.05 0.312      1000      1000
## 
## Method: Generalized Estimating Equation 
## 
## Variance Parameters:
##      sigma_b_sq
## Arm1      0.024
## Arm2      0.021
## 
## Clusters:
##      n.clust
## Arm1       6
## Arm2       6
## 
## Observations:
## $Arm1
## [1] 50 50 50 50 50 50
## 
## $Arm2
## [1] 50 50 50 50 50 50
## 
## 
## Convergence:
## TRUE 
## 1000
\end{verbatim}

This simulation suggests that our power would be between 0.66-0.71 --
i.e., moderately powered.

\hypertarget{gee-model-assuming-or0.60}{%
\subsection{GEE Model Assuming
OR=0.60}\label{gee-model-assuming-or0.60}}

\begin{Shaded}
\begin{Highlighting}[]
\CommentTok{# geesim4 <- cps.binary( nsim = 1000,}
\CommentTok{#                        nsubjects = 50, # 50 participants per cluster assuming equal sizes}
\CommentTok{#                        nclusters = 6, # 6 clusters per treatment arm}
\CommentTok{#                        p1=.40, # 40% non-abstinence rate in TAU in Maricich et al. (2020)}
\CommentTok{#                        p2=.29, # 25% non-abstinence rate in reSET-0 group in Maricich et al. (2020)}
\CommentTok{#                        sigma_b_sq=0.024, # variance = p(1-p)=0.24; assume 10% variance is between)}
\CommentTok{#                        sigma_b_sq2=0.021, # variance = p(1-p)=0.21; assume 10% variance is between)}
\CommentTok{#                        alpha=.05,}
\CommentTok{#                        method='gee',}
\CommentTok{#                        quiet=F,}
\CommentTok{#                        seed=0218211,}
\CommentTok{#                        lowPowerOverride = TRUE )}
\KeywordTok{load}\NormalTok{( }\StringTok{'geesim4.RData'}\NormalTok{ )}
\KeywordTok{summary}\NormalTok{( geesim4 )}
\end{Highlighting}
\end{Shaded}

\begin{verbatim}
## 
## Monte Carlo Power Estimation based on 1000 Simulations: Simple Design, Binary Outcome. Note: 0 additional models were fitted to account for non-convergent simulations.
## 
## Power Estimate (alpha = 0.05):
##  Power Lower.95.CI Upper.95.CI Alpha  Beta Converged Requested
##  0.763   0.7353915   0.7890505  0.05 0.237      1000      1000
## 
## Method: Generalized Estimating Equation 
## 
## Variance Parameters:
##      sigma_b_sq
## Arm1      0.024
## Arm2      0.021
## 
## Clusters:
##      n.clust
## Arm1       6
## Arm2       6
## 
## Observations:
## $Arm1
## [1] 50 50 50 50 50 50
## 
## $Arm2
## [1] 50 50 50 50 50 50
## 
## 
## Convergence:
## TRUE 
## 1000
\end{verbatim}

This simulation suggests that our power would be between 0.74-0.79 --
i.e., approaching the goal of 0.80.

\hypertarget{cox-propotional-hazards-model}{%
\section{Cox Propotional Hazards
Model}\label{cox-propotional-hazards-model}}

\end{document}
